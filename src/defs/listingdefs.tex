\usepackage{listingsutf8}
\usepackage[dvipsnames]{xcolor}
\usepackage{tcolorbox}
\tcbuselibrary{listings}
\tcbuselibrary{skins}

\colorlet{listingbg}{Goldenrod!10}

\lstdefinestyle{common}{
    language=Python,
    basicstyle=\ttfamily,
    keywordstyle=\color{blue},
    commentstyle=\color{OliveGreen},
    stringstyle=\color{BrickRed},
    showstringspaces=true,
}

\lstdefinestyle{colored}{
    style=common,
    backgroundcolor=\color{listingbg},
}

\lstdefinestyle{numbers}{
    style=common,
    numbers=left,
    stepnumber=1,
    numberstyle=\tiny,
    numbersep=10pt
}

\lstdefinestyle{nonumbers}{
    style=common,
    numbers=none
}

\tcbset{listingboxcommon/.style={%
    enhanced,
    boxrule=0.2pt,
    frame style={draw=black!10},
    colback=listingbg,
    coltitle=black!80,
    colbacktitle=black!10,
    fonttitle=\bfseries,
    listing only,
    listing options={
        style=common,
    },
}}

\tcbset{numbers/.style={%
    listingboxcommon,
    listing options={
        style=numbers,
        xleftmargin=1em,
    },
}}

\tcbset{nonumbers/.style={%
    listingboxcommon,
    listing options={
        style=nonumbers,
    },
}}

\NewTCBInputListing{\inputsrcfile}{ O{} D||{} m !O{0-1000} }{%
    % #1: tcolorbox options
    % #2: listing options
    % #3: file name
    % #4: line range
    listing file=#3,
    listing options={
        style=common,
        linerange={#4},
        #2,
    },
    numbers,
    #1,
}

\NewDocumentCommand{\code}{ m }{
    \hspace{-.5em}
    \lstinline[style=nonumbers]|#1|
    \hspace{-.5em}
}
